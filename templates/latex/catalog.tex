% generated with model catalog generator.
% Use KOMA-Script article class
\documentclass[pdftex,a4paper,12pt,titlepage]{scrartcl}
% packages
\usepackage[utf8]{inputenc}
\usepackage[T1]{fontenc}
\usepackage{scrlayer-scrpage}
\usepackage[table]{xcolor}
\usepackage[default]{sourcesanspro}
\usepackage[left=1.5cm,right=1cm,top=1.5cm,bottom=2.5cm]{geometry}
\usepackage[hidelinks]{hyperref}
\usepackage{multicol}
\usepackage{graphicx}
\usepackage{color}
\usepackage{tabularx}
\usepackage{mathptmx}
\usepackage{fontsize}
\usepackage{supertabular}
\usepackage{needspace}
% Set defauilt font to helvetica too.
\cehead*{\VAR{ title|escape_latex }}
\cohead*{\VAR{ title|escape_latex }}
\cefoot*{Lucky Resistor – https://luckyresistor.me/}
\cofoot*{Lucky Resistor – https://luckyresistor.me/}
\lefoot*{\pagemark}
\rofoot*{\pagemark}
\pagestyle{scrheadings}
\pagenumbering{arabic}
\setcounter{tocdepth}{\subsectiontocdepth}
\renewcommand{\familydefault}{\sfdefault}
\renewcommand{\arraystretch}{1.5}
\definecolor{row1}{gray}{0.95}
\definecolor{row2}{gray}{0.9}
% Title page.
\title{Catalog}
\subtitle{\VAR{ title|escape_latex }}
\author{\href{https://luckyresistor.me}{Lucky Resistor – https://luckyresistor.me/}}
\begin{document}
\begin{titlepage}
\begin{center}
\vspace*{15mm}
{\bfseries\fontsize{65}{75}\selectfont Catalog \par}
\vspace*{5mm}
{\bfseries\fontsize{40}{50}\selectfont \VAR{ title|escape_latex } \par}
\vspace*{1cm}
\includegraphics[width=\columnwidth]{\VAR{ compressed_images[title_image] }}
\vspace*{1mm}
{\bfseries\fontsize{25}{30}\selectfont Lucky Resistor \\ \href{https://luckyresistor.me}{https://luckyresistor.me/}}
\end{center}
\end{titlepage}
\tableofcontents

\pagebreak
\section{Print Recommendations}

\begin{center}
\rowcolors{2}{row1}{row2}
\begin{tabularx}{\columnwidth}{ l X }
    \rowcolor{black!60}\textcolor{white}{\textbf{Parameter}} & \textcolor{white}{\textbf{Recommendation}} \\
\BLOCK{ for r in recommendations }\textbf{\VAR{ r[0]|escape_latex }} & \VAR{ r[1]|escape_latex } \\
\BLOCK{ endfor }
\end{tabularx}
\end{center}

\BLOCK{ if drawing_image }
\section{Dimensions}

\begin{center}
\includegraphics[width=\columnwidth]{\VAR{ compressed_images[drawing_image] }}
\end{center}
\BLOCK{ endif }

\vfill

\pagebreak
\section{Models}\label{models}

\BLOCK{ for model_group in model_groups }
\BLOCK{ if not loop.first }\pagebreak\BLOCK{ endif }
\BLOCK{ if model_group.title }\subsection{\VAR{ model_group.title|escape_latex }}\BLOCK{ endif }

\begin{multicols}{2}
\BLOCK{ for model in model_group.models }
\BLOCK{ if model_group.title }\subsubsection\BLOCK{ else }\subsection\BLOCK{ endif }{
    \BLOCK{if model.title}\VAR{model.title}\BLOCK{else}\VAR{model.part_id}\BLOCK{endif}}
\hypertarget{\VAR{ model.part_id|escape_latex }}{}

\begin{minipage}{\columnwidth}
\footnotesize
\includegraphics[width=\columnwidth]{\VAR{ compressed_images[model.image_files[0].name] }}

\rowcolors{2}{row1}{row2}
\begin{tabularx}{\columnwidth}{ l X }
    \rowcolor{black!60}\textcolor{white}{\textbf{Parameter}} & \textcolor{white}{\textbf{Value}} \\
\BLOCK{ for p in parameter }\textbf{\VAR{ p.title|escape_latex }} & \VAR{model.formatted_values[p.name]|escape_latex} \\
\BLOCK{ endfor }
\textbf{Model Files} & \BLOCK{ for name in model.model_files }\BLOCK{ if not loop.first }\newline \BLOCK{ endif }\VAR{name|escape_latex}\BLOCK{ endfor } \\
\end{tabularx}
\end{minipage}

\vfill

\BLOCK{ endfor }
\end{multicols}
\BLOCK{ endfor }

\BLOCK{ for table_group in table_groups }

\pagebreak
\section{\VAR{ table_group.title|escape_latex }}

\BLOCK{ if table_columns > 1 }\begin{multicols}{\VAR{ table_columns }}\BLOCK{ endif }
\BLOCK{ for table in table_group.tables }

\Needspace{5\baselineskip}
\subsection{\VAR{ table.title|escape_latex }}
\tablehead{ \rowcolor{black!60}\BLOCK{ for title in table.fields }\textcolor{white}{\textbf{\VAR{ title|escape_latex }}} \BLOCK{ if loop.last }\\ \BLOCK{ else }& \BLOCK{ endif }\BLOCK{ endfor } }
\rowcolors{2}{row1}{row2}
\begin{supertabular}{ \BLOCK{ for _ in table.fields }l \BLOCK{ endfor } }
\BLOCK{ for row in table.rows }\hyperlink{\VAR{ row[0]|escape_latex }}{\textbf{\VAR{ row[0]|escape_latex }}} \BLOCK{ for value in row }\BLOCK{ if not loop.first }& \VAR{ value|escape_latex } \BLOCK{ endif }\BLOCK{ endfor }\\
\BLOCK{ endfor }
\end{supertabular}

\BLOCK{ endfor }
\BLOCK{ if table_columns > 1 }\end{multicols}\BLOCK{ endif }
\BLOCK{ endfor }

\end{document}


